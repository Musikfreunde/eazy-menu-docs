Die Oberösterreichische Versicherung benutzt ein veraltetes Bestellsystem.
Die vorliegende Diplomarbeit beschäftigt sich mit einer Kantinenwebapp und einem zusätzlichen Android Client, die zur Essensbestellung
verwendet werden. 
\\*
Ziel dieser Diplomarbeit ist es allen Mitarbeitern der Oberösterreichischen Versicherung ein Benutzerfreundliches Bestellsystem
bereitzustellen und somit auf keine veraltete Technologie beschränkt zu sein. 

Die Webapp bietet zwei Arten von Benutzern an.
Ein Kantinen-Benutzer, welcher die einzelnen Menüs erstellt und verarbeitet
und noch ein Mitarbeiter-Benutzer, der die jeweiligen Menüs bestellen kann.
Die Daten für die beiden Benutzerinterfaces liefert das Quarkus-Backend zusammen mit einer Oracle Datenbank.
 Durch die Statistikansicht hat der Mitarbeiter eine
übersicht seiner Bestellungen. Dem Benutzer wird aber auch für die einzelnen Tagen Menüs vorgeschlagen
welche anhand dem Bestellverlauf ermittelt werden. Die einzelnen Menüs werden von der Kantine in verschiedene Kategorien kategorisiert.
Somit kann man leicht sein passendes Menü finden. 