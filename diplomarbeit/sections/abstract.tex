\begin{spacing}{1}
    \chapter*{Abstract}
\end{spacing}
\begin{wrapfigure}{r}{0.3\textwidth}
    \begin{center}
      \includegraphics[width=0.2\textwidth]{pics/question_mark.png}
    \end{center}
\end{wrapfigure}
EazyMenu is an application for managing food orders in company canteens and was developed to replace an application of OÖV based on IBM Notes. 
This application was only accessible on the desktop and did not meet today's standards. 
EazyMenu instead allows employees to order a meal quickly and easily via their mobile phone or from a desktop. 
Meals can be categorized by tags. An algorithm then uses the tags to decide what is the best meal for an employee.
The cafeteria also has the ability to create new meals and manage them as well. 
It is also possible to print out an order overview with the most important information for a particular day.


\newpage

\begin{spacing}{1}
    \chapter*{Zusammenfassung}
\end{spacing}
\begin{wrapfigure}{r}{0.3\textwidth}
    \begin{center}
      \includegraphics[width=0.2\textwidth]{pics/question_mark.png}
    \end{center}
\end{wrapfigure}
\author{David Ignjatovic}
EazyMenu ist eine Anwendung zur Verwaltung von Essensbestellungen in Firmenkantinen und wurde entwickelt, um eine Altapplikation der OÖV auf Basis von IBM Notes abzulösen. 
Diese Altapplikation war nur auf dem Desktop zugänglich und entsprach nicht den heutigen Standards. 
Im Gegensatz dazu ermöglicht EazyMenu Mitarbeiterinnen und Mitarbeitern per Handy oder von einem Desktop aus schnell und einfach eine Mahlzeit zu bestellen. 
Dabei können Gerichte durch Tags kategorisiert werden. Ein Algorithmus entscheidet dann anhand der Tags was die beste Mahlzeit für eine Mitarbeiterin oder einem Mitarbeiter ist.
Auch die Kantine hat die Möglichkeit neue Mahlzeiten zu erstellen und diese auch zu verwalten. 
Es besteht auch die Möglichkeit eine Bestellübersicht mit den wichtigsten Informationen für einen bestimmten Tag auszudrucken.