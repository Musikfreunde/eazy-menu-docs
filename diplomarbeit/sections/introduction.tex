\section{Ausgangsituation}
Die Oberösterreichische Versicherung ist der Marktführer im Versicherungsbereich in Oberösterreich 
und beschäftigt in ihrer Zentrale in Linz über 500 Mitarbeiter.
Das Unternehmen besitzt eine Kantine, wo täglich 3 Hauptspeisen serviert werden. Dazu noch
eine Vorspeise und eine Nachspeise. 
\section{Ist-Zustand}
Die derzeitige Bestellmöglichkeit funktioniert über eine simple Datenbankanwendung mit IBM Notes.
Dieses Programm ist auf jedem Rechner installiert und jeder Mitarbeiter ist mit seinen Daten bereits eingeloggt.
Auf einem simplen Interface kann man zwischen den heutigen und zukünftigen Mahlzeiten wählen und die Uhrzeit, wann man 
diese konsumiert. Nach erfolgreicher Bestellung hat man eine kleine Übersicht über die vergangenen Bestellungen.
\section{Problemstellung}
Das derzeitige Programm ist sehr veraltet und nicht besonders benutzerfreundlich. Außerdem
läuft das Programm lokal auf jedem Rechner und eine mobile Bestellung ist ausgeschlossen.
Außerdem ist zu erwähnen ist, dass der Benutzer nur eine beschränkte Möglichkeit hat sein Bestellverhalten 
zu visualisieren bzw. zu analysieren.
\section{Aufgabenstellung}
Die Aufgabenstellung der Firma ist eine Webanwendung zu entwickeln, die den Vorgänger ablöst
und ein modernes und benutzerfreundliches Interface hat. Zudem soll eine Bestellung über das Smartphone
möglich sein. Zusätzlich soll ein Empfehlungssytem entwickelt werden, dass einem ermöglicht Mahlzeiten zu 
bestellen, die auf einen abgestimmt sind. Außerdem soll man eine Einsicht über seine Bestellhistorie haben
mit diversen Statistikelementen.
\section{Ziel(e)}
\begin{itemize}
    \item Erleichterung des Bestellprozesses für den Benutzer
    \item Flexible Bestellungsmöglichkeiten
    \item Der Benutzer hat mehr Verständnis über sein Bestellverhalten
\end{itemize}

\subsection{Zielgruppe}
Mitarbeiter der OÖ Versicherung AG
