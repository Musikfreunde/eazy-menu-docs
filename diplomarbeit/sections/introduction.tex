\section{Ausgangsituation}
\author{Benjamin Besic}
Die Oberösterreichische Versicherung ist der Marktführer im Versicherungsbereich in Oberösterreich 
und beschäftigt in ihrer Zentrale in Linz über 500 Mitarbeiter.
Das Unternehmen besitzt eine Kantine, wo täglich 3 Hauptspeisen serviert werden. Dazu noch
eine Vorspeise und eine Nachspeise. 
\section{Ist-Zustand}
\author{Benjamin Besic}
Die derzeitige Bestellmöglichkeit funktioniert über eine simple Datenbankanwendung mit IBM Notes.
Dieses Programm ist auf jedem Rechner installiert und jeder Mitarbeiter ist mit seinen Daten bereits eingeloggt.
Auf einem simplen Interface kann man zwischen den heutigen und zukünftigen Mahlzeiten wählen und die Uhrzeit, wann man 
diese konsumiert. Nach erfolgreicher Bestellung hat man eine kleine Übersicht über die vergangenen Bestellungen.
\section{Problemstellung}
\author{Benjamin Besic}
Das derzeitige Programm ist sehr veraltet und nicht besonders benutzerfreundlich. Außerdem
läuft das Programm lokal auf jedem Rechner und eine mobile Bestellung ist ausgeschlossen.
Außerdem ist zu erwähnen ist, dass der Benutzer nur eine beschränkte Möglichkeit hat sein Bestellverhalten 
zu visualisieren bzw. zu analysieren.
\section{Aufgabenstellung}
\author{Benjamin Besic}
Die Aufgabenstellung der Firma ist eine Webanwendung zu entwickeln, die den Vorgänger ablöst
und ein modernes und benutzerfreundliches Interface hat. Zudem soll eine Bestellung über das Smartphone
möglich sein. Zusätzlich soll ein Empfehlungssytem entwickelt werden, dass einem ermöglicht Mahlzeiten zu 
bestellen, die auf einen abgestimmt sind. Außerdem soll man eine Einsicht über seine Bestellhistorie haben
mit diversen Statistikelementen.
\section{Ziel(e)}
\author{Benjamin Besic}
\begin{itemize}
    \item Erleichterung des Bestellprozesses für den Benutzer
    \item Flexible Bestellungsmöglichkeiten
    \item Der Benutzer hat mehr Verständnis über sein Bestellverhalten
\end{itemize}

\subsection{Zielgruppe}
Mitarbeiter der OÖ Versicherung AG

\section{Use Cases}

\subsection{Kantinenarbeiter}

\begin{compactitem}
    \item Neue Menüs anlegen
    \begin{compactitem}
        \item Ein Kantinenmitarbeiter kann für jeden Tag neue Menüs mit drei Hauptspeisen und deren Kategorien, einer Vorspeise und einer Nachspeise anlegen.
    \end{compactitem}
    \item Vorhandene Menüs editieren
    \begin{compactitem}
        \item Die Bezeichnungen der bereits erstellten Menüs sollen verändert werden können.
    \end{compactitem}
    \item Übersicht der täglichen Bestellungen
    \begin{compactitem}
        \item Die Kantinenmitarbeiter sollen eine Übersicht, der an einem bestimmten Tag bestellten Menüs haben. Diese inkludiert die zusammengefasste Bestellanzahl der verschiedenen Menüs und eine Liste von allen Bestellungen.
    \end{compactitem}
    \item Bestellungsübersicht drucken
    \begin{compactitem}
        \item Die Übersicht wie vorher beschrieben soll zu einem PDF-Objekt konvertiert werden und dementsprechend ausgedruckt werden können.
    \end{compactitem}
\end{compactitem}

\subsection{Mitarbeiter}

\begin{compactitem}
    \item Menüs bestellen
    \begin{compactitem}
        \item Ein Mitarbeiter hat eine Auswahl aller Menüs und kann für jeden Tag eine der drei Hauptspeisen auswählen. Nach der Auswahl kann er die Essenszeit auswählen, die Anzahl und nötige Kommentare hinzufügen.
    \end{compactitem}
    \item Menüs für andere Mitarbeiter bestellen
    \begin{compactitem}
        \item Ein Mitarbeiter kann den obrigen Bestellvorgang für einen anderen Mitarbeiter ausführen. 
    \end{compactitem}
    \item Übersicht aller Bestellungen
    \begin{compactitem}
        \item Als Mitarbeiter soll man alle seine vergangenen Bestellungen und deren Informationen in einer Übersicht einsehen können. Diese Übersicht kann filtriert werden.
    \end{compactitem}
    \item Bestellungen stornieren
    \begin{compactitem}
        \item In der oben genannten Übersicht soll man die Möglichkeit haben eine Bestellung auszuwählen und zu stornieren, wenn dies möglich ist.
    \end{compactitem}
    \item Bestellstatistiken einsehen
    \begin{compactitem}
        \item Ein Mitarbeiter soll Diagramme zur Verfügung haben, wo er sein Bestellverhalten einsehen kann.
    \end{compactitem}
\end{compactitem}
