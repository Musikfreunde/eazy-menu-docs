\section{IntelliJ IDEA}

IntelliJ IDEA ist eine der führenden Entwicklungsumgebungen für die Programmiersprache Java. Sie wurde vom Unternehmen Jetbrains im Jahre 2000 entwickelt.
Außerdem bietet sie ebenfalls Entwicklungsmöglichkeiten für Kotlin, Groovy, Scala und auch Android.
Sie ist immer auf dem neusten Entwicklungsstand, wird laufend mit Updates versorgt und unterstützt die derzeit gängigen Programmiertools
wie Docker, Kubernetes, Maven, Datenbank-Tools, Git, Jakarta EE und viele weitere.
Es gibt eine kostenpflichtige Ultimate Version und eine Community Version, die kostenfrei zur Verfügung gestellt wird.
IntelliJ zeichnet auch die Anzahl an Erweiterungen mittels Plugins aus. Die Umgebung besitzt auch eine
sehr intuitive Intelligenz, die es dem Entwickler sehr einfach macht damit zu programmieren.
\cite{IntJ} \\*
Wir haben uns dafür entschieden, da wir damit viel Erfahrung hatten und die oben genannten Punkte
unterstützten unsere Entscheidung enorm.

\section{Git}
\cite{GitKinsta}


\section{Java}

\section{Java EE}

\subsection{Java EE vs. Quarkus}

\subsection{JPA}

\subsection{Hibernate}

\subsection{Panache}

\section{JBoss}

\section{Cypress}

\section{Keycloak}

\section{Oracle Datenbank}

\section{Vue.js}
\author{}

\subsection{Progressive Web App(PWA)}


\subsection{Json Web Token (JWT)}

\subsection{Google Charts}

\section{Javascript}

\section{Maven}

\subsubsection{Deepest}
Vermeide mich.