\section{Protokolle}
\begin{table}
    
    \begin{tabular}{|p{3cm}|p{10cm}|  }
        \hline
        Datum & 25-10-2021\\
        \hline
        Anwesende & Benjamin Besic, Bozidar Spasenovic, David Ignjatovic\\
        \hline
        Zwischenstand&  Das Programm ist lauffähig und erfüllt alle usecases. \\
        \hline
        Bis zum nächsten mal &  Schriftliche Diplomarbeit\\
        \hline
        
    \end{tabular}
    \caption{Protokoll}
    \label{tab:my_label}
\end{table}
\begin{table}
    \begin{tabular}{ |p{3cm}|p{10cm}|   }
        \hline
        Datum & 12-11-2021\\
        \hline
        Anwesende & Benjamin Besic, Bozidar Spasenovic, David Ignjatovic\\
        \hline
        Zwischenstand& Diagramme wurden gezeigt mit dummy werten.\\
        \hline
        Bis zum nächsten mal &  Food Recommender, Android, Schriftlicher Teil \\
        \hline
    \end{tabular}
    \caption{Protokoll}
    \label{tab:my_label}
\end{table}
\begin{table}
    \begin{tabular}{ |p{3cm}|p{10cm}|  }
        \hline
        Datum & 26-11-2021\\
        \hline
        Anwesende & Benjamin Besic, Bozidar Spasenovic, David Ignjatovic\\

        \hline
        Zwischenstand&  erste Version Android, Recommender, schriftlicher Teil
    
    \\
        \hline
        Bis zum nächsten mal &  

       Android: restlichen Fragmente,logik
        

    frontend mit backend verbinden

    grafiken mit neuem backend

    statistiken

    versuchen den schriftlichen Teil für einen ersten Überblick fertigstellen


    
    \\
        \hline
    \end{tabular}
    \caption{Protokoll}
    \label{tab:my_label}
\end{table}
\begin{table}
    \begin{tabular}{ |p{3cm}|p{10cm}|  }
        \hline
        Datum & 10-12-2021\\
        \hline
        Anwesende & Benjamin Besic, Bozidar Spasenovic, David Ignjatovic\\

        \hline
        Zwischenstand&  Jetpack compose

        Es wurde das neue Projekt gezeigt, welches von XML auf Jetpack compose umgeschrieben.
        Die App ist zurzeit mit statischen werten befüllt.
        Problem ist das Jetpack compose relativ neu ist, und wenig informationsquellen online zur verfügung stehen.
        1.2. Backend
        
        Backend ist im groben und ganzen fertig. Logic für die Statistiken wird noch erledigt.
        1.3. Recommender
        
        Frontend wurde fertiggestellt für den Recommender.
        \\
        \hline
        Bis zum nächsten mal &  



        Statistik (Backend/Frontend) wird fertiggestellt

        Android App wird erweitert
    
    
    
    \\
        \hline
    \end{tabular}
    \caption{Protokoll}
    \label{tab:my_label}
\end{table}
\begin{table}
    \begin{tabular}{ |p{3cm}|p{10cm}|   }
        \hline
        Datum & 06-01-2022\\
        \hline
        Anwesende & Benjamin Besic, Bozidar Spasenovic, David Ignjatovic\\

        \hline
        Zwischenstand& Es wurde das neue Projekt gezeigt, welches von XML auf Jetpack compose umgeschrieben.
        Die App ist zurzeit mit statischen werten befüllt.
        Problem ist das Jetpack compose relativ neu ist, und wenig informationsquellen online zur verfügung stehen.
        
        Backend ist im groben und ganzen fertig. Logic für die Statistiken wird noch erledigt.
        
        Frontend wurde fertiggestellt für den Recommender.\\
        \hline
        Bis zum nächsten mal &  





        Android app muss noch fertig gemacht werden. (Spasenovic)

        KeyCloak mit JetPackCompose
    
        Statistiken anzeigen
    
        Recommender und Statistiken evt. erweitern (Ignjatovic und Besic)
    
    


    
    \\
        \hline
    \end{tabular}
    \caption{Protokoll}
    \label{tab:my_label}
\end{table}
\begin{table}
    \begin{tabular}{ |p{3cm}|p{10cm}|  }
        \hline
        Datum & 28-01-2022\\
        \hline
        Anwesende & Benjamin Besic, Bozidar Spasenovic, David Ignjatovic\\

        \hline
        Zwischenstand& 

        KeyCloak funktioniert jetzt
    
        Bestellungen werden angezeigt
    
   Probleme
    
        Beim Bestellen kommt ein fehler (unsupported mediatype).
    
        Login button muss 2 mal gedrückt werden.
    
            möglicher Fehler: api läuft asynchron
    
        hartkodierte urls vermeiden
    
    \\
        \hline
        Bis zum nächsten mal &  

        Schriftlich weiterschreiben / Beisc, Ignjatovic (evt. Spasenovic)
    
            verwendete Technologien
    
            implementierung
    
        Android Projekt verschönern (Spasenovic)
    
        Automatisierte Test
    \\
        \hline
    \end{tabular}
    \caption{Protokoll}
    \label{tab:my_label}
\end{table}
\begin{table}
    \begin{tabular}{ |p{3cm}|p{10cm}|  }
        \hline
        Datum & 04-02-2022\\
        \hline
        Anwesende & Benjamin Besic, Bozidar Spasenovic, David Ignjatovic\\

        \hline
        Zwischenstand& 

        Was funktioniert:

    Login noch nicht ganz fertig - KeyCloak fehlt

    Logik hinter der App funktioniert

    Was geht leider noch nicht

    Post funktioniert nicht ganz

    Statistiken in der Android App gehen nicht




    
    \\
        \hline
        Bis zum nächsten mal & 





        Kotlin Login verbessern / Spasenovic

        Backend Bug fixen / Ignjatovic
    
        schriftlicher Teil (Implementierung) / evt (alle)
    
    


    
    
    
    \\
        \hline
    \end{tabular}
    \caption{Protokoll}
    \label{tab:my_label}
\end{table}
\begin{table}
    \begin{tabular}{ |p{3cm}|p{10cm}|  }
        \hline
        Datum & 23-02-2022\\
        \hline
        Anwesende & Benjamin Besic, Bozidar Spasenovic, David Ignjatovic\\

        \hline
        Zwischenstand& Was geht nicht

        Probleme mit dem formbody in postBestellung
    
            403 MediaType not supported
    
    Was fehlt noch
    
        Bestellung
    
        eventuelle verschönerungen
    
     Was wurde gemacht
    
        Login gefixt mit CountDownLatch()
    
        Verlauf
    
    \\
        \hline
        Bis zum nächsten mal &  

        Schriftlicher Teil / alle
    
            Implementierung / Verwendete Technologien
    
        Bestellung fixen / Spasenovic
    
            Android
    
        Unit Test / Ignjatovic
    
            Backend
    
    \\
        \hline
    \end{tabular}
    \caption{Protokoll}
    \label{tab:my_label}
\end{table}




\begin{table}
    \begin{tabular}{ |p{3cm}|p{10cm}|   }
        \hline
        Datum & 17-03-2022\\
        \hline
        Anwesende & Benjamin Besic, Bozidar Spasenovic, David Ignjatovic\\

        \hline
        Zwischenstand& 



        bestellung funktioniert jetzt

        fehler war falscher name des Atributes

        personalnummer und orderFor war null



    \\
        \hline
        Bis zum nächsten mal &  Android

        bessere package namen verwenden
    
        Ip Addressen nicht hardcoden
    
     Schriftlich
     Planung
    
        liegt verloren
    
        sagt wenig aus
    
        Datenmodel Diagramme
    
        Datenmodel Diagramme zur Implementierung verschieben.
    
        Eventuell Planung zur Implementierung.
    
        Mockups erweitern mit Text.
    
        UseCases zur Implementierung
    
     Backend
    
        Swagger wäre nicht schlecht und dann im schriftlichen Teil einbauen
    
        Im schriftlichen Teil
    
            Weniger Code bei attribute
    
    \\
        \hline
    \end{tabular}
    \caption{Protokoll}
    \label{tab:my_label}
\end{table}
\begin{table}
    \begin{tabular}{ |p{3cm}|p{10cm}|   }
        \hline
        Datum & 25-03-2022\\
        \hline
        Anwesende & Benjamin Besic, Bozidar Spasenovic, David Ignjatovic\\

        \hline
        Zwischenstand& 

        Request in eigen File geschrieben
    
        Component im schriftlichen Teil beschrieben
    
    \\
        \hline
        Bis zum nächsten mal & 

        abstract
    
        zusammenfassung
    
        hinten ausführliche zusammenfassung könnte mehrere Seiten haben
    
     \\
        \hline
    \end{tabular}
    \caption{Protokoll}
    \label{tab:my_label}
\end{table}
\begin{table}
    \begin{tabular}{ |p{3cm}|p{10cm}|   }
        \hline
        Datum & 01-04-2022\\
        \hline
        Anwesende & Benjamin Besic, Bozidar Spasenovic, David Ignjatovic\\

        \hline
        Zwischenstand& 

        Github Actions/Packages geschrieben
    
        Implementierung geschrieben
    
    \\
        \hline
        Bis zum nächsten mal &  
    
    \\
        \hline
    \end{tabular}
    \caption{Protokoll}
    \label{tab:my_label}
\end{table}